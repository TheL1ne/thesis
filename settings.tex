%%
% Table of Contents
% 1. Declaration of Variables
% 2. Developer Settings
% 3. USER SETTINGS
% 4. Macros
% 5. Environments
%%

%%
% 1. Declaration of Variables (don't change this!)
%%
\newboolean{isPrintMode}
\newboolean{isContentOnlyMode}
\newboolean{showExampleChapter}
\newboolean{showNotes}
\newboolean{showAbstract}
\newboolean{customChapterStyle}
\newboolean{showCustomHeaderAndFooter}
\newboolean{showPlainHeaderAndFooter}
\newboolean{showDefaultHeaderAndFooter}%this is only a dummy item
\newboolean{showTableOfContents}
\newboolean{showListOfTables}
\newboolean{showListOfAcronyms}
\newboolean{showGlossary}
\newboolean{showListOfFigures}
\newboolean{showStatutoryDeclaration}
\newboolean{addEmptyPageToEnd}
%\newboolean{showMultipleAbstracts}


%%
% 2. Developer Settings
%%
\newcommand{ \chapterPath }{content/}
\newcommand{ \frontmatterPath }{\chapterPath}
\newcommand{ \statutoryDeclarationPath }{\chapterPath}
\newcommand{ \miscPath }{misc/}
%\newcommand{ \srcPath }{sources/} %doesn't work at the moment
\newcommand{ \titlePath }{titlepages/}

%%
% 3. User Settings (customize it until it fits your needs!)
%%
\newcommand{ \titleType }{default}

\newcommand{ \degree }{Bachelor}
\newcommand{ \thesisType }{\degree arbeit}
\newcommand{ \thesisTitle }{Regelbasierte Anomalieerkennung mit Hilfe von Ontologien}
\newcommand{ \submissionDate }{\today}
\newcommand{ \place }{Dresden}
\newcommand{ \institution }{Technische Universität Dresden}
\newcommand{ \department }{Fakultät Informatik}
\newcommand{ \subject }{Professur für Prozesskommunikation}
\newcommand{ \country }{Deutschland}
\newcommand{ \firstName }{Benjamin}
\newcommand{ \lastName }{Riedel}
\newcommand{ \name }{\firstName\ \lastName}
\newcommand{ \studentnumber }{3576322}
\newcommand{ \responsiblePerson }{Dr. Ing. Robert Lehmann}
\newcommand{ \supervisor }{Nico Braunisch}
\newcommand{ \reviewer }{Prof. Dr. Ing. habil Wollschlaeger}

\newcommand{ \citeStyle }{numeric}

% images
\newcommand{ \logoFileName }{TUD_Logo_HKS41_228}

\newcommand{ \numberOfChapters }{6}

% cleans up the document for printing e.g. hides examples
\setboolean{isPrintMode}{false}
  \setboolean{showExampleChapter}{true}%only appears if isPrintMode is set to false
  \setboolean{showNotes}{true}%only appears if isPrintMode is set to false
% hides all pages except the chapters and following
\setboolean{isContentOnlyMode}{false}% is being ignored in printMode

% which parts of a book should be included? select them
\setboolean{showAbstract}{true}
%  \setboolean{showMultipleAbstracts}{true}
\setboolean{showTableOfContents}{true}
\setboolean{showListOfTables}{false}
\setboolean{showListOfAcronyms}{true}
\setboolean{showGlossary}{true}
\setboolean{showListOfFigures}{false}
\setboolean{showStatutoryDeclaration}{true}
\setboolean{addEmptyPageToEnd}{false}

%% Only set one of 'em to true!
\setboolean{showDefaultHeaderAndFooter}{true}
\setboolean{showPlainHeaderAndFooter}{false}
\setboolean{showCustomHeaderAndFooter}{false}%see directory >>Misc<<

\setboolean{customChapterStyle}{false}%customize your chapter style in misc/chapterStyle

%%
% 4. Macros
%%
\newcommand{ \getChapterPath }[1]{\chapterPath chapter#1}
\newcommand{ \getRunningElementPath }[1]{\miscPath #1}
\newcommand{ \getTitlePath }[1]{\titlePath #1}

%to-do macros TODO and FIXME and TIP
\newcounter{todoCOUNT}
\newcounter{fixmeCOUNT}
\newcounter{tipCOUNT}
\newcommand{\TODO}[1]{
  \ifisPrintMode
  \else
    \vspace{0.5em}\todo[inline, color=orange]{#1}\stepcounter{todoCOUNT}
  \fi
}

\newcommand{\FIXME}[1]{
  \ifisPrintMode
  \else
    \todo[size=\small, color=red]{#1}\stepcounter{fixmeCOUNT}
  \fi
}

\newcommand{\TIP}[1]{
  \ifisPrintMode
  \else
    \todo[size=\small, color=yellow]{#1}\stepcounter{tipCOUNT}
  \fi
}

%%
% 5. Environments
%%
\newcommand\abstractname{Abstract}  %%% here
\makeatletter

\newenvironment{abstract}{%
    \if@twocolumn
      \section*{\abstractname}%
    \else
      \small
      \begin{center}%
        {\bfseries \abstractname\vspace{-.5em}\vspace{\z@}}%
      \end{center}%
      \quotation
    \fi}
    {\if@twocolumn\else\endquotation\fi}
\makeatother
