\chapter{Einleitung}
Informationssysteme waren lange Zeit von der der realen Welt entkoppelt und folgten deterministischen, monolithischen Abläufen. Doch Miniaturisierung ist in der Industrie auf dem Vormarsch und mit ihr immer komplexere und flexiblere Automationsabläufe. Die Anzahl der Teilnehmer an einem Industrienetzwerk haben sich in den letzten Jahren dafür drastisch erhöht und dabei noch weiter verzweigt [Zitat]. Auch die Netzwerkstruktur hat sich der Notwendigkeit der komplexeren Verteilung angepasst, indem sie sich von einer Daisy-Chain, über das BUS-System und Sternarchitektur zu hierarchischen oder Meshnetzwerken entwickelt hat. [Zitat]
Gründe für diese Veränderung sind neben der Miniaturisierung von Produktionsanlagen und dem verstärkten Einsatz von Robotern für ehemals manuelle Aufgaben auch der Einsatz von neuen Technologien wie „Augmented Reality“ [Zitat], „Cyber-Physical-Systems“ oder der Erlaubnis private Geräte mit in bestimmte Netzwerke einwählen zu dürfen („Bring your own device“, BYOD, [Zitat]).
Doch was passiert, wenn ein Netzwerkteilnehmer plötzlich die Nutzbarkeit der Infrastruktur torpediert? Wie bemerkt man frühzeitig, dass sich eine Flut von Nachrichten durch das Netzwerk ausbreiten, weil alle Geräte immer wieder versuchen ihre Nachrichten zu verschicken oder dass ein Gerät plötzlich Kontakt zu Teilen des Netzwerks aufnimmt mit denen es nicht kommunizieren sollte? Oder was ist, wenn sich ein Gerät einfach aufhängt? Oder wenn sich eine Fräse plötzlich außerhalb des Sicherheitsbereichs eingeschaltet hat?
Das frühzeitige Erkennen von solchem Fehlverhalten fällt in die Disziplin der Anomalieerkennung. Dieses Forschungsfeld ist von besonderem Interesse, da Missstände in IT-Systemen neben Menschenleben auch schnell viele hunderttausende Euro kosten können. Leider treten immer wieder Fälle auf bei denen Automationslösungen zu Gefahr für Leib, Leben und Ausrüstung werden [Zitat].

