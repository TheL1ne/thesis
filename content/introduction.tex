\chapter{Einleitung}
Aktuelle Trends in der Automatisierung deuten auf ein Umdenken hin wie Produktionslinien zukünftig organisiert werden sollen. So sind in den letzten 10 Jahren neue Buzzwords wie \acrfull{cps}, \acrfull{iot} und \Gls{Industrie40} entstanden. Hinter den drei Begriffen verbergen sich unterschiedliche Foki in Forschungsarbeiten mit einer Gemeinsamkeit: sie begünstigen eine verteilte Struktur von Automatisierungslösungen und damit auch eine erhöhte Komplexität der Netzwerkstruktur. Ziel dieser neuen Organisation ist eine flexiblere Produktion bei gleichzeitig besserer Auslastung der verfügbaren Ressourcen im System und damit verbunden eine bessere Energieeffizienz.\cite{wollschlaeger2017future} \\
Außerdem gibt es Bestrebungen die Arbeitskraft der Mitarbeiter besser zu kanalisieren und beispielsweise Einarbeitungszeit in neue Geräte zu Minimieren und im gleichen Zug die Zufriedenheit der Mitarbeiter mit dem Arbeitsgerät zu erhöhen. Was liegt also näher als den Angestellten einfach sein eigenes Gerät (Smartphone, Laptop oder Tablet) nutzen zu lassen und dieses in die Unternehmensumgebung zu integrieren? Die Vorteile liegen auf der Hand: zufriedenere, effektivere Angestellte und weniger Kosten für das Unternehmen. Auch in kritischen Umgebungen wie Krankenhäusern wird darüber gesprochen solche Strategien zu adaptieren. \cite{wani2020hospital, french2014current} \\
Eine andere Richtung der Forschung ist die Realisierung von echt-zeitnahen Netzwerken auf günstiger Ethernethardware durch Erweiterungen im Ethernetprotokoll. Sie entspringt der Tatsache, dass Unterhaltungselektronik mit dem Aufkommen von Streamingdiensten ebenfalls höhere Anforderungen an die Stabilität und Echtzeitnähe stellt\cite{wollschlaeger2017future} und damit die Qualität der verfügbaren Hardware steigt. So waren in der Vergangenheit die realisierbaren Bandbreiten und die Konstanz der Verbindung gering im Vergleich mit industriellen Technologien wie Profibus \cite{tovar1999real}, da es in vielen Situationen unproblematisch war, wenn hoher \Gls{Jitter} im Endkundennetzwerk aufgetreten ist. Doch das hat sich geändert. Dadurch ist es heute verlockend weiche Echtzeitanforderungen mit Hilfe von günstigen Ethernet-Geräten umzusetzen. Denn durch die Massenhafte Nutzung ist die Anschaffung solcher Geräte deutlich günstiger \cite{enduserSwitch, profinetSwitch}. All diese unterschiedlichen Veränderungen in industriellen Netzwerken und wie diese konzipiert sind, führen zu komplexeren und vielschichtigen Netzen in industriellen Anlagen. \\
Doch diese neuen Bestrebungen haben auch Nachteile: so sind Netzwerke, die sich über größere Strecken verteilen und weniger strikt konfigurierte Teilnehmern haben, naturgemäß angreifbarer und störanfälliger als Netzwerke die sich nahezu gänzlich in einem Gebäude befinden, nur limitierte Schnittstellen zu öffentlichen Netzwerken haben und denen striktere Konfigurationen zu Grunde liegen. Allerdings soll es um die Unterscheidung dieser Anforderungen und Netzwerke hier nicht gehen. Es macht aber deutlich wie vielfältig die aktuelle Situation in der Industrie ist. Daher muss bei der Konzeption von Netzwerken aktuell besonders viel Wert auf die Erkennung von Störungen geworfen werden. Aber bei der hohen Komplexität ist es nicht mehr möglich Störungen rein manuell zu erkennen und automatisierte Lösungsansätze sind notwendig. In den letzten Jahren wurden dabei viele Ansätze verfolgt und evaluiert. In dieser Arbeit wird es um einen neuen Ansatz basierend auf \Glspl{Ontologie} gehen.\\
So gliedert sich die Arbeit folgender Maßen: nach dieser Einleitung finden sich das Ziel und die Motivation der Arbeit und welche Vorteile wir uns von dem Ansatz versprechen. In Kapitel 3 besprechen wir eine Übersicht von Arbeiten die sich mit Anomalieerkennung in industriellen Netzwerken befassen. Außerdem wird auch ein Einblick in andere Anwendungsgebiete von \Glspl{Ontologie} gegeben, insbesondere wo \Glspl{Ontologie} zur Anomalieerkennung in Datensätzen genutzt werden. Kapitel 4 beschreibt den Versuchsaufbaues zur Evaluation des Ansatzes und welche Limitierungen damit einhergehen. In Kapitel 5 werden die gewonnen Daten dann präsentiert und evaluiert. Schließlich findet sich im letzten Kapitel noch eine Zusammenfassung der Arbeit mit Anregung für weitere Forschungsarbeiten.