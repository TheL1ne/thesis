\chapter{Einleitung}
Aktuelle Trends in der Automatisierung deuten auf ein Umdenken in der Art und Weise wie Produktionslinien zukünftig organisiert werden sollen hin. So sind in den letzten 10 Jahren neue Buzzwords entstanden, allen voran \acrfull{cps}, \acrfull{iot} und \Gls{Industrie40}. Alle Forschungsbereiche teilen eine Gemeinsamkeit: sie begünstigen eine verteilte Struktur von Automatisierungslösungen und damit auch eine erhöhte Komplexität der Netzwerkstruktur. Ziel dieser neuen Organisation ist eine flexiblere Produktion bei gleichzeitig besserer Auslastung der verfügbaren Ressourcen im System und damit verbunden eine bessere Energieeffizienz. \\
Doch diese neuen Bestrebungen haben auch Nachteile: so sind Netzwerke, die sich über größere Strecken verteilen, naturgemäß angreifbarer und störanfälliger als Netzwerke die sich nahezu gänzlich in einem Gebäude befinden und nur limitierte Schnittstellen zu öffentlichen Netzwerken haben. Doch mit der steigenden Komplexität der Netzwerke ist es nicht mehr Möglich Störungen rein manuell zu erkennen und automatisierte Lösungsansätze sind notwendig. Insbesondere Ansätze, die schnell und zuverlässig möglichst viele Anomalien erkennen können, die viel Geld kosten können, ob nun durch physische Schäden an Maschinen oder kopierte Firmeninterna durch Angreifer. \\
Im Gegensatz zu bisherigen Ansätzen, die Prozesse auf deterministische, finite Automaten abbildeten, soll es in dieser Arbeit um einen Ansatz basierend auf \Glspl{Ontologie} gehen. So gliedert sich die Arbeit folgender Maßen: nach dieser Einleitung finden sich die Vorteile gegenüber den bisherigen Ansätzen, die \Glspl{Ontologie} mit sich bringen, und eine detaillierte Beschreibung des Ziels dieser Arbeit. In Kapitel 3 findet sich dann eine Übersicht von Arbeiten die sich mit Anomalieerkennung und industriellen Netzwerken befassen. Dabei wird auch ein Einblick in andere Anwendungsgebiete gegeben, in denen bereits \Glspl{Ontologie} zur Anomalieerkennung in Datensätzen zur genutzt werden. Kapitel 4 beschreibt dann den Versuchsaufbaues zur Evaluation des Ansatzes und welche Limitierungen damit einhergehen. In Kapitel 5 werden die gewonnen Daten dann präsentiert und evaluiert. Schließlich findet sich im letzten Kapitel noch eine Zusammenfassung der Arbeit.