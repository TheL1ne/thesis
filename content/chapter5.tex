\chapter{Evaluation}
Es ist uns nicht gelungen eine einzige Anomalie zu erkennen. So war die Art und Weise, wie die eingesetzte Bibliothek zum Aufzeichnen von Netzwerkverkehr, die einzelnen Pakete verarbeitet, unvorteilhaft. Denn es war schwierig möglich die Auzeichnung von Paketen nach einer festen Zeit zu unterbrechen, trotz eines vorhandenen Timeout-Parameters. Denn dieser wird laut der Dokumentation\cite{pcaploop} beim Mitschnitt von Paketen nicht ausgewertet und die gesammelte Erfahrung im Umgang mit der Bibliothek stützt das. So blieb das Programm trotz eines zehn-Sekunden-Timeouts mehrfach über Minuten in einem Wartezustand hängen. In anderen Fällen war es dem Programm nicht möglich die gesendeten Netzwerkpakete mitzuschneiden, obwohl nach einem einfachen Neustart keine Probleme auftraten. Es scheint hier Zuverlässigkeitsprobleme durch die Bibliotheken zu geben. Es kann nicht ausgeschlossen werden, dass es zu einer fehlerhaften Nutzung der bereitgestellten Klassen durch uns kam. Dem gegenüber steht aber die fehlerfreie Funktion nach einem Neustart.\\
Dazu kommt die Schnellebigkeit von \textit{Open-Source-Projekten}. So sind die \textit{OWL-API}, die Bibliotheken zum Aufzeichnen von Netzwerkpaketen und viele \Gls{reasoner} als solche Projekte organisiert. So wurde einer der bekanntesten \Gls{reasoner} \textit{Pellet}\cite{pellet} seit 2017 nicht mehr aktualisiert und wurde dadurch nicht auf die 2016 erschienen Version fünf der \textit{OWL-API} aktualisiert. Das verringert die Attraktivität von diesen Lösungen, da es so keine Optimierungen und Anpassungen an aktuelle Problemstellungen gibt.
\section{Limitierungen des Versuchsaufbaus}
Leider war es uns durch die angewandten Technologien zur Umsetzung des \Gls{mockup}s nicht möglich eine größere Abweichung der Paketgröße zu simulieren. So konnten an die definierten Pakete nicht einfach weitere Bytes angehängt werden um eventuelle Rechteausweitung, boshaftes Verhalten oder Fehler in dieser Form zu simulieren. Außerdem war es aufgrund des synchronen Charakters von \textit{gRPC}\cite{grpc} nicht möglich Bestätigungen zu verwehren. Durch die Anwendungen von \textit{Protobuf}\cite{protobuf} als Serialisierungsebene konnten auch keine leeren Pakete geschickt werden, da die Kodierungsimplementation dies nicht zu lies. Diese Anomalien müssen in der Zukunft evaluiert werden. Des weiteren war es uns auf Grund von Zeitmangel nicht möglich einen effektiven Vergleichsalgorithmus für die einzelnen Zustände zu implementieren.

\section{Probleme des Ansatzes}
Der Entwurf einen korrekten und widerspruchsfreien Ontologie nimmt viel Zeit in Anspruch. Da \Glspl{axiom} zu Schlussfolgerungen bei der Evaluation führen und so weitere \Glspl{axiom} entstehen, kommt es bereits beim Erstellen von Ontologien zu einem enormen Ingenieursaufwand. Diese Arbeit wird durch den Umstand erschwert, dass \Gls{reasoner}, durch die subsumierten Schlussfolgerungen, ein nicht verursachendes Axiom als widersprüchlich kennzeichnen können. Insbesondere bei komplexen und vielschichtigen Netzwerkstrukturen ist mit sehr hohem Zeitaufwand zu rechnen. Dieser Aufwand kann bei breiter Anwendung von \Glspl{Ontologie} allerdings entgegen gewirkt werden, da eine Wiederverwendung von vorhanden Definitionen möglich ist\cite{borst1999construction}. Des Weiteren ist es notwendig für jedes Netzwerk individuellen Code zu schreiben um Beobachtungen auf \Glspl{axiom} abzubilden. Dies ist offensichtlich mit weiterem hohen Aufwand verbunden. Hier ist ebenfalls zu erwarten, dass sich \textit{Designpattern}\cite{proulx2000programming} entwickeln. Damit würde sich die benötigte Zeit zum Erstellen des passenden Verbindungscodes reduzieren lassen, aber zum aktuellen Zeitpunkt existiert diese Option nicht.\\
Außerdem scheint Netzwerkanalyse basierend auf Java in den letzten Jahren ein Gebiet mit schwindendem Interesse zu sein. So sind viele verlinkte Websiten offline und trotz komplexer Abhängigkeiten der vorhanden Bibliotheken, wurde keine Aktualisierung für \textit{Maven}\cite{maven} vorgenommen. Zusätzlich kann man anhand des Codeanteils im genutzten \textit{Github}-Repository\cite{gitcode} gut erkennen, dass Java nur als Abstraktionsebene genutzt wird und ein Großteil der genutzten Bibliotheken auf \textit{C} zurückgreift. Nach den Erfahrungen der Autoren dieser Arbeit ist die Familie der \textit{C}-Sprachen die übliche Wahl, wenn man Netzwerkpakete auf Bitebene analysieren muss.\\
Des Weiteren legen Geschwindigkeitsevaluationen Nahe, dass komplexe Ontologiemodelle oder zu vielen aktiven Individuen, die klassifiziert werden müssen, eine stärker als lineare Steigerung der benötigten Zeit bis zur fertigen Klassifizierung bedeutet. Insbesondere bei starker Nutzung von \textit{ODER}-Verknüpfungen\cite[p.~5]{roy2010exploitation}. Das kann die Nutzung dieses Ansatzes zu Echtzeitanomalieerkennung bei großen Netzwerken verhindern.\\

\section{Vorteile des Ansatzes}
Von den erhofften Vorteilen aus \autoref{chap:2:advantages} ist nur die Aussicht auf eine Vereinheitlichung der Wissensstruktur einer Domäne geblieben. Manche Arbeiten sehen darin ein aktuelles Bedürfnis und eine Notwendigkeit um verteilte, smarte und flexible Fabriken in Zukunft umzusetzen. Besonders im Sinne der \Gls{cps} ist das für Anwender erstrebenswert, um eine Vereinheitlichung der Kommunikationschnittstellen zwischen den Herstellern erreichen zu können. Damit wäre es möglich selbst organisierende Herstellungsketten einfacher zu realisieren. \cite{lasi2014industry, kolberg2015lean}