\newglossaryentry{Anomalie}
{
  name={Anomalie},
  description={ist ein wahrgenommenes, abnormales Verhalten eines Systems oder eines Teil eines Systems \cite{pardo2016framework},
  plural={Anomalien}
}

\newglossaryentry{Industrie40}
{
  name={Industrie 4.0},
  description={ist eine Bestrebung von mehreren Unternehmen zu Digitalisierung der industriellen Produktion und damit verbundener Automation. Dabei wird auf Techniken aus den Feldern \acrshort{cps} und \acrshort{iot} zurück gegriffen}
}

\newglossaryentry{Ontologie}
{
  name={Ontologie},
  description={ist eine "explizite Spezifikation eines Konzepts". \cite{guarino2009ontology} Dabei besteht das Konzept aus Klassen und deren Beziehungen zu einander und wird dann auf spezifische Individuen angewandt um sie entsprechenden Klassen zuzuordnen. Eine Ontologie dient so dazu komplexe Sachverhalte in einer Kombination von einfacheren Aussagen darzustellen, TODO: ontologie definiton nachschlagen},
  plural={Ontologien}
}

\newglossaryentry{Jitter}
{
  name={Jitter},
  description={bildet die Varianz oder den Unterschied in der Zeitdauer ab, die Pakete in einem Netzwerk zwischen zwei Knoten benötigen. Es ist damit eine Variable zur Darstellung der Stabilität einer Netzwerkverbindung},
}

\newglossaryentry{Roundtriptime}
{
  name={Round-Trip Time},
  description={bezeichnet die Zeitspanne zwischen Absenden einen Datenpakets und der Bestätigung durch den Empfänger},
}

\newglossaryentry{M2M}
{
  name={M2M},
  description={Die Abkürzung rührt aus dem englischen \textit{Machine to Machine}, dass sich in dieser Arbeit wiederum auf eine direkte Kommunikation zwischen zwei Maschinen/Anlagen bezieht},
}

\newglossaryentry{gracefulshutdown}
{
  name={Graceful Shutdown},
  description={bezeichnet eine Art des Beenden einer Netzwerkressource. Dabei wird der Ressource ein Terminierungssignal gesendet und der Ressource danach noch Zeit eingeräumt eventuelle Aufräum- oder Abspeicherungsvorgänge zu beenden und sich dann selbst herunter zu fahren},
}

\newglossaryentry{rollingupdates}
{
  name={Rolling Updates},
  description={wird eine Strategie genannt, die dazu dient Netzwerkservices zu aktualisieren. Dabei wird erst eine aktualisierte Version des Services hochgefahren und neue Anfragen dahin umgeleitet, bevor der alte Service herunter gefahren wird. Besonders bei Cloudanbietern ist dies üblich.},
}

\newglossaryentry{batch}
{
  name={Batch},
  description={wird eine Sammlung von gleichförmigen Objekten in Informationstechnologie genannt. So werden beispielsweise mehrere Einzelabfragen in einem Paket an eine Datenbank gegeben. So ein Paket wird Batch genannt},
  plural={Batches}
}

\newglossaryentry{label}
{
  name={Label},
  description={bezieht sich in dieser Arbeit auf die Einstufung eines Zustandes als Anomalie oder nicht. Ein Anomalieerkennungsalgorithmus kann als Ergebnis ein Label ausgeben, dass entweder das Vorhandensein einer Anomalie oder den Sollzustand markiert},
  plural={Labels}
}

\newglossaryentry{score}
{
  name={Score},
  description={ist eine Bewertung. In dieser Arbeit bezieht sich ein Score auf die Wahrscheinlichkeit einer Anomalie. Der Wertebereich liegt zwischen 0 und 1},
  plural={Scores}
}

\newglossaryentry{signatur}
{
  name={Signatur},
  description={ist eine Sammlung von Merkmalen, die es ermöglicht wichtige Unterscheidungen zu Treffen. So erkennen wir anhand von Netzwerkmerkmalen (wie IP, Zeichenfolgen in Paketen, Anfragenabfolge, Abfragegeschwindigkeit und -häufigkeit usw.) Angriffe \cite{singh2012detecting},
  plural={Signaturen}
}

\newglossaryentry{falsepositive}
{
  name={False-Positive},
  description={tritt auf wenn ein Anomalieerkennungsalgorithmus eine normale Anfrage als Anomalie einstuft},
  plural={False-Positives}
}