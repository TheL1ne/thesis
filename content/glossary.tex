\newglossaryentry{Industrie40}
{
  name={Industrie 4.0},
  description={Industrie 4.0 ist eine Bestrebung von mehreren Unternehmen zu Digitalisierung der industriellen Produktion und damit verbundener Automation. Dabei wird auf Techniken aus den Feldern \acrshort{cps} und \acrshort{iot} zurück gegriffen}
}

\newglossaryentry{Ontologie}
{
  name={Ontologie},
  description={Eine Ontologie ist eine "explizite Spezifikation eines Konzepts". \cite{guarino2009ontology} Dabei besteht das Konzept aus Klassen und deren Beziehungen zu einander und wird dann auf spezifische Individuen angewandt um sie entsprechenden Klassen zuzuordnen. Eine Ontologie dient so dazu komplexe Sachverhalte in einer Kombination von einfacheren Aussagen darzustellen},
  plural={Ontologien}
}

\newglossaryentry{Jitter}
{
  name={Jitter},
  description={Jitter bildet die Varianz oder den Unterschied in der Zeitdauer ab, die Pakete in einem Netzwerk zwischen zwei Knoten benötigen. Es ist damit eine Variable zur Darstellung der Stabilität einer Netzwerkverbindung},
}

\newglossaryentry{Roundtriptime}
{
  name={Round-Trip Time},
  description={Der Sender überträgt ein Datenpaket über das Netzwerk zum Empfänger, wenn der Empfänger das Paket erhält, schickt der Empfänger eine Bestätigung an den Sender. Die zwischen Absender und Erhalt der Bestätigung wird \textit{round-trip time} genannt},
}

\newglossaryentry{M2M}
{
  name={M2M},
  description={Die Abkürzung rührt aus dem englischen \textit{Machine to Machine}, dass sich in dieser Arbeit wiederum auf eine direkte Kommunikation zwischen zwei Maschinen/Anlagen bezieht},
}

\newglossaryentry{gracefulshutdown}
{
  name={graceful shutdown},
  description={bezeichnet eine Art des Beenden einer Netzwerkressource. Dabei wird der Ressource ein Terminierungssignal gesendet und der Ressource danach noch Zeit eingeräumt eventuelle Aufräum- oder Abspeicherungsvorgänge zu beenden und sich dann selbst herunter zu fahren},
}

\newglossaryentry{rollingupdates}
{
  name={rolling updates},
  description={ist eine Strategie um Services zu aktualisieren. Dabei wird erst eine aktualisierte Version des Services hochgefahren und neue Anfragen dahin umgeleitet, bevor der alte Service herunter gefahren wird. Besonders bei Cloudanbietern ist dies üblich.},
}