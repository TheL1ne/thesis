\newglossaryentry{Anomalie}
{
  name={Anomalie},
  description={ist ein wahrgenommenes, abnormales Verhalten eines Systems oder eines Teil eines Systems \cite{pardo2016framework}},
  plural={Anomalien}
}

\newglossaryentry{Industrie40}
{
  name={Industrie 4.0},
  description={ist eine Bestrebung von mehreren Unternehmen zu Digitalisierung der industriellen Produktion und damit verbundener Automation. Dabei wird auf Techniken aus den Feldern \acrshort{cps} und \acrshort{iot} zurück gegriffen}
}

\newglossaryentry{Ontologie}
{
  name={Ontologie},
  description={ist eine \textit{formelle Spezifikation einer geteilten Begriffsbildung}\cite{guarino2009ontology}. Dabei bestehen die definierten Begriffe aus Konzepten und deren Beziehungen zu einander und werden dann auf spezifische Individuen angewandt um sie entsprechenden Konzepten zuzuordnen. Eine Ontologie dient so dazu komplexe Sachverhalte in einer Kombination von einfacheren Aussagen darzustellen oder zu beschreiben. Sie entspringen dem Forschungsfeld der \textit{Beschreibungslogik}\cite{roy2010exploitation}},
  plural={Ontologien}
}

\newglossaryentry{Jitter}
{
  name={Jitter},
  description={bildet die Varianz oder den Unterschied in der Zeitdauer ab, die Pakete in einem Netzwerk zwischen zwei Knoten für den Transport benötigen. Es ist damit eine Variable zur Darstellung der Stabilität einer Netzwerkverbindung},
}

\newglossaryentry{Roundtriptime}
{
  name={Round-Trip Time},
  description={bezeichnet die Zeitspanne zwischen Absenden einen Datenpakets und der Bestätigung durch den Empfänger},
}

\newglossaryentry{M2M}
{
  name={M2M},
  description={Die Abkürzung rührt aus dem englischen \textit{Machine to Machine}. Diese bezieht sich in dieser Arbeit auf eine direkte Kommunikation zwischen zwei Maschinen/Anlagen},
}

\newglossaryentry{gracefulshutdown}
{
  name={Graceful Shutdown},
  description={bezeichnet eine Art des Beendens einer Netzwerkressource. Dabei wird der Ressource ein Terminierungssignal gesendet und ihr danach noch Zeit eingeräumt eventuelle Aufräum- oder Abspeicherungsvorgänge zu beenden und sich dann selbst herunter zu fahren},
}

\newglossaryentry{rollingupdates}
{
  name={Rolling Updates},
  description={wird eine Strategie genannt, die dazu dient Netzwerkservices zu aktualisieren. Dabei wird erst eine aktualisierte Version des Services hochgefahren und neue Anfragen dahin umgeleitet, bevor der alte Service herunter gefahren wird. Besonders bei Cloudanbietern ist dies üblich.},
}

\newglossaryentry{batch}
{
  name={Batch},
  description={wird eine Sammlung von gleichförmigen Objekten in Informationstechnologie genannt. So werden beispielsweise mehrere Einzelabfragen in einem Paket an eine Datenbank gegeben. So ein Paket wird Batch genannt},
  plural={Batches}
}

\newglossaryentry{label}
{
  name={Label},
  description={bezeichnet einen binären Wert, der anzeigt ob eine Anomalie im bewerteten Fall vorhanden ist. Es ist eine von zwei Arten wie Anomalieerkennungsalgorithmen ihre Ergebnisse ausgeben können\cite{ahmed2016survey}},
  plural={Labels}
}

\newglossaryentry{score}
{
  name={Score},
  description={ist eine Bewertung. In dieser Arbeit bezieht sich ein Score auf die Wahrscheinlichkeit einer Anomalie. Der Wertebereich liegt zwischen 0 und 1},
  plural={Scores}
}

\newglossaryentry{signatur}
{
  name={Signatur},
  description={ist eine Sammlung von Merkmalen, die es ermöglicht wichtige Unterscheidungen zu treffen. So erkennen wir anhand von Netzwerkmerkmalen (wie IP, Zeichenfolgen in Paketen, Anfragenabfolge, Abfragegeschwindigkeit und -häufigkeit usw.) Angriffe \cite{singh2012detecting}},
  plural={Signaturen}
}

\newglossaryentry{falsepositive}
{
  name={False-Positive},
  description={tritt auf wenn ein Anomalieerkennungsalgorithmus eine normale Anfrage als Anomalie einstuft},
  plural={False-Positives}
}

\newglossaryentry{rdf}
{
  name={Ressource-Description-Framework},
  description={ist ein Framework zur Anreicherung von Webdokumenten mit Metainformationen. Ziel ist es Maschinen das Auswerten von Informationen im Web zu ermöglichen um die Effizienz der Nutzer bei der Bewegung im \textit{World Wide Web} zu erhöhen},
}

\newglossaryentry{reasoner}
{
  name={Reasoner},
  description={ist die Realisierung einer Inferenzmaschine und kann zum Schlussfolgern über Ontologien genutzt werden. Die Hauptoperation ist dabei das Zusammenfassen von Klassen (engl \textit{to subsume}},
}

\newglossaryentry{mockup}
{
  name={Mockup},
  description={(oder kurz Mock)ist eine Abstraktion eines Programms oder Services, die den Platz einer vollständigen Implementierung in einer Modellanwendung einnehmen kann. Ein Mockup dient der Veranschaulichung einer Funktionsweise. Häufig werden Mocks in Testumgebungen (zum Beispiel bei Integrationstests) eingesetzt},
  plural={Mockups}
}

\newglossaryentry{konzept}
{
  name={Konzept},
  description={bezeichnet in einer Ontologie eine definierte Klasse, über die Aussagen getroffen werden},
  plural={Konzepte}
}

\newglossaryentry{axiom}
{
  name={Axiom},
  description={sind Aussagen in einer Ontologie und bilden damit die Grundlage von Ontologien},
  plural={Axiome}
}
