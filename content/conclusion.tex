\chapter{Zusammenfassung}
Wir wollten in dieser Arbeit zeigen, dass es möglich ist mit Hilfe von \Glspl{Ontologie} regelbasiert Anomalien in einem Netzwerkstack zu erkennen. Auf Grund dem Umstand, dass \Glspl{Ontologie} auch zur Repräsentation von Wissen genutzt werden können, wäre das ein hervorragender Gewinn besonders in den Bestrebungen um \Gls{Industrie40}. Aber durch technischen Schwierigkeiten, Zeitmangel und die Limitierung durch die Ontologie-Community auf Java, war es uns nicht möglich diesen Beweis zu erbringen. Der aufwendigste Schritt ist dabei die Zusammenführung von \Gls{Ontologie} als Soll-Zustands-Modell und die Abbildung der aufgezeichneten Pakete auf dieses Modell. Durch die hohe manuelle Arbeit raten wir aktuell davon ab den hier gezeigten Ansatz in einem Produktionssystem nutzen zu wollen. Bereits für unser reduziertes Modell aus \autoref{chap:4:model} war es sehr aufwendig und erforderte mehrere Tage Arbeit, die Abbildung der aufgezeichneten Pakete auf Klassen und Relationen im \Gls{Ontologie}-Modell zu erarbeiten.\\
Sollte der hier gezeigte Ansatz trotzdem wieder aufgegriffen werden, wäre eine Anwendung auf Netzwerkanomaliedatensätze wie \textit{DARPA}\cite{darpa99}, \textit{KDD}\cite{kdd99} oder \textit{ADFA-LD12}\cite{adfald} ein besserer Ansatz, um die Ergebnisse mit anderen Algorithmen vergleichen zu können. Allerdings bedarf es dazu einer vorherigen Analyse welche Nutzwerkstruktur diesen Datensätzen zu Grunde liegt. Denn im Gegensatz zu Ansätzen wie dem \textit{Chi-Square}-Verfahren\cite{ye2001anomaly} oder dem \textit{Clustering}\cite{likas2003global} ist bei diesem Ansatz deutlich mehr Ingeneursaufwand notwendig bevor das System erstmalig gestartet werden kann.